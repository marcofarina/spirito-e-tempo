% Options for packages loaded elsewhere
\PassOptionsToPackage{unicode}{hyperref}
\PassOptionsToPackage{hyphens}{url}
%
\documentclass[
]{book}
\title{Il Libro dello Spirito e del Tempo}
\author{Prof.~Marco Farina}
\date{2022-03-23}

\usepackage{amsmath,amssymb}
\usepackage{lmodern}
\usepackage{iftex}
\ifPDFTeX
  \usepackage[T1]{fontenc}
  \usepackage[utf8]{inputenc}
  \usepackage{textcomp} % provide euro and other symbols
\else % if luatex or xetex
  \usepackage{unicode-math}
  \defaultfontfeatures{Scale=MatchLowercase}
  \defaultfontfeatures[\rmfamily]{Ligatures=TeX,Scale=1}
\fi
% Use upquote if available, for straight quotes in verbatim environments
\IfFileExists{upquote.sty}{\usepackage{upquote}}{}
\IfFileExists{microtype.sty}{% use microtype if available
  \usepackage[]{microtype}
  \UseMicrotypeSet[protrusion]{basicmath} % disable protrusion for tt fonts
}{}
\makeatletter
\@ifundefined{KOMAClassName}{% if non-KOMA class
  \IfFileExists{parskip.sty}{%
    \usepackage{parskip}
  }{% else
    \setlength{\parindent}{0pt}
    \setlength{\parskip}{6pt plus 2pt minus 1pt}}
}{% if KOMA class
  \KOMAoptions{parskip=half}}
\makeatother
\usepackage{xcolor}
\IfFileExists{xurl.sty}{\usepackage{xurl}}{} % add URL line breaks if available
\IfFileExists{bookmark.sty}{\usepackage{bookmark}}{\usepackage{hyperref}}
\hypersetup{
  pdftitle={Il Libro dello Spirito e del Tempo},
  pdfauthor={Prof.~Marco Farina},
  hidelinks,
  pdfcreator={LaTeX via pandoc}}
\urlstyle{same} % disable monospaced font for URLs
\usepackage{longtable,booktabs,array}
\usepackage{calc} % for calculating minipage widths
% Correct order of tables after \paragraph or \subparagraph
\usepackage{etoolbox}
\makeatletter
\patchcmd\longtable{\par}{\if@noskipsec\mbox{}\fi\par}{}{}
\makeatother
% Allow footnotes in longtable head/foot
\IfFileExists{footnotehyper.sty}{\usepackage{footnotehyper}}{\usepackage{footnote}}
\makesavenoteenv{longtable}
\usepackage{graphicx}
\makeatletter
\def\maxwidth{\ifdim\Gin@nat@width>\linewidth\linewidth\else\Gin@nat@width\fi}
\def\maxheight{\ifdim\Gin@nat@height>\textheight\textheight\else\Gin@nat@height\fi}
\makeatother
% Scale images if necessary, so that they will not overflow the page
% margins by default, and it is still possible to overwrite the defaults
% using explicit options in \includegraphics[width, height, ...]{}
\setkeys{Gin}{width=\maxwidth,height=\maxheight,keepaspectratio}
% Set default figure placement to htbp
\makeatletter
\def\fps@figure{htbp}
\makeatother
\setlength{\emergencystretch}{3em} % prevent overfull lines
\providecommand{\tightlist}{%
  \setlength{\itemsep}{0pt}\setlength{\parskip}{0pt}}
\setcounter{secnumdepth}{5}
\usepackage{booktabs}
\usepackage{amsthm}
\makeatletter
\def\thm@space@setup{%
  \thm@preskip=8pt plus 2pt minus 4pt
  \thm@postskip=\thm@preskip
}
\makeatother
\ifLuaTeX
  \usepackage{selnolig}  % disable illegal ligatures
\fi
\usepackage[]{natbib}
\bibliographystyle{apalike}

\begin{document}
\maketitle

{
\setcounter{tocdepth}{1}
\tableofcontents
}
\hypertarget{introduzione}{%
\chapter{Introduzione}\label{introduzione}}

\emph{Il Libro dello Spirito e del Tempo} è un percorso in cui vi potrete allenare per diventare dei \emph{Super Sayan} del codice. Dico ``codice'' in generale perché questo libro non è pensato per essere risolto in nessun particolare linguaggio di programmazione. Ai miei studenti verrà indicato o consigliato quale usare, o vi sarà lasciata libera scelta a seconda di cosa avete voglia di approfondire.

Il capitolo \ref{fondamenti} racchiude gli esercizi basilari su \textbf{input} e \textbf{output}, sulla gestione delle \textbf{variabili} e su semplici istruzioni di \textbf{calcolo} usando le operazioni aritmetiche fondamentali.

\hypertarget{fondamenti}{%
\chapter{Fondamenti di programmazione}\label{fondamenti}}

\hypertarget{input-output-e-semplici-istruzioni}{%
\section{Input, output e semplici istruzioni}\label{input-output-e-semplici-istruzioni}}

\hypertarget{hello-world}{%
\subsection{Hello World!}\label{hello-world}}

\hypertarget{obiettivo}{%
\subsubsection*{Obiettivo}\label{obiettivo}}
\addcontentsline{toc}{subsubsection}{Obiettivo}

Il primissimo programma che ogni informatico realizza è il famoso ``Hello World!''. Realizzare un programma che stampa a video la stringa \texttt{Hello\ World!}.

\begin{center}\rule{0.5\linewidth}{0.5pt}\end{center}

\hypertarget{greetings}{%
\subsection{Greetings}\label{greetings}}

\hypertarget{obiettivo-1}{%
\subsubsection*{Obiettivo}\label{obiettivo-1}}
\addcontentsline{toc}{subsubsection}{Obiettivo}

Scrivere un programma che usa la funzione di input per chiedere all'utente il suo nome. Visualizzare in output un messaggio di benvenuto che includa il nome dell'utente.

\hypertarget{esempio}{%
\subsubsection*{Esempio}\label{esempio}}
\addcontentsline{toc}{subsubsection}{Esempio}

\begin{verbatim}
Come ti chiami?
> Mario
Benvenuto, Mario! Sei pronto per salvare la Principessa Peach?
\end{verbatim}

\begin{center}\rule{0.5\linewidth}{0.5pt}\end{center}

\hypertarget{il-triangolo-no}{%
\subsection{Il triangolo no}\label{il-triangolo-no}}

\hypertarget{obiettivo-2}{%
\subsubsection*{Obiettivo}\label{obiettivo-2}}
\addcontentsline{toc}{subsubsection}{Obiettivo}

Scrivere un programma che legge in input tre numeri interi maggiori di zero e calcola, visualizzandoli in output, il perimetro e l'area.

\hypertarget{esempio-1}{%
\subsubsection*{Esempio}\label{esempio-1}}
\addcontentsline{toc}{subsubsection}{Esempio}

\begin{verbatim}
Lato 1:
> 3
Lato 2:
> 5
Lato 3:
> 7
Il perimetro è lungo 15.
L'area è 
\end{verbatim}

\hypertarget{hint}{%
\subsubsection*{Hint}\label{hint}}
\addcontentsline{toc}{subsubsection}{Hint}

Per calcolare l'area di un triangolo di cui sono note le misure dei lati si può usare la \textbf{formula di Erone}. Indicando con \(a\), \(b\) e \(c\) i lati di un triangolo qualsiasi, si calcola prima il semipreimetro

\[
p = \frac{a+b+c}{2}
\]
A questo punto l'area si calcola tramite la formula
\[
A = \sqrt{p(p-a)(p-b)(p-c)}
\]

  \bibliography{book.bib,packages.bib}

\end{document}
